
\section{Introduction}
    Our project is a \todo. This section will describe this document and provide the glossary/references.
    \subsection{Document Purpose}
        \emph{Identify the product whose software requirements are specified in this document, including the revision or release number. Describe the scope of the product that is covered by this SRS, particularly if this SRS describes only part of the system or a single subsystem.\gnl Write 1-2 paragraphs describing the purpose of this document as explained above.}
    \subsection{Product Scope}
        \emph{Provide a short description of the software being specified and its purpose, including relevant benefits, objectives, and goals.\gnl 1-2 paragraphs describing the scope of the product. Make sure to describe the benefits associated with the product.}
    \subsection{Intended Audience and Document Overview}
        \emph{Describe the different types of reader that the document is intended for, such as developers, project managers, marketing staff, users, testers, and documentation writers (In your case it would probably be the `client' and the professor). Describe what the rest of this SRS contains and how it is organized. Suggest a sequence for reading the document, beginning with the overview sections and proceeding through the sections that are most pertinent to each reader type.}
        There are 3 main groups of users for \projectName:
        \begin{enumerate}
            \item \emph{Music schools}. Music schools manage the instructors and students under their umbrella of operation. They have permissions to add/remove association with the other user types, as well as special operations regarding interactions between the user groups.
            \item \emph{Music instructors}. Instructors interact with their associated companies and students. The students can be private or managed by a music school. Instructors are primarily responsible for lesson plans and practice resources for their students, and sharing such resources with schools or other teachers.
            \item \emph{Music students}. Students have access to a history of their lesson plans, practice resources, and communication with their associated instructor or school. 
        \end{enumerate}
    \subsection{Definitions}
        \printnoidxglossary
        \printnoidxglossary[type=acronym, title=Acronyms and Abbreviations]
        \printacronyms
    \vspace{-4ex} % Get rid of extra space after the acronyms
    \subsection{Document Conventions}
        In general this document follows the IEEE formatting requirements.
        % Use Arial font size 11, or 12 throughout the document for text. Use italics for comments. Document text should be single spaced and maintain the 1” margins found in this template. For Section and Subsection titles please follow the template.\gnl Describe any standards or typographical conventions that were followed when writing this SRS, such as fonts or highlighting that have special significance. Sometimes, it is useful to divide this section to several sections, e.g., Formatting Conventions, Naming Conventions, etc.}
        % \subsubsection{Formatting Conventions}
            \begin{itemize}
                \item Arial font family
                \item 12pt font
                \item Italicized comments
                \item Single spaced lines
                \item $1\inch$ margins
            \end{itemize}
        % \subsubsection{Naming}
        %     \begin{itemize}
        %         \item 
        %     \end{itemize}
    \subsection{References and Acknowledgements}
        % Print all entries in bib file for testing
        \nocite{*}
        \printbibliography[heading=none]
