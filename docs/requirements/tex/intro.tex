
\section{Introduction}
    \emph{Please provide a brief introduction to your project and a brief overview of what the reader will find in this section.}
    \subsection{Document Purpose}
        \emph{Identify the product whose software requirements are specified in this document, including the revision or release number. Describe the scope of the product that is covered by this SRS, particularly if this SRS describes only part of the system or a single subsystem.\gnl Write 1-2 paragraphs describing the purpose of this document as explained above.}
    \subsection{Product Scope}
        \emph{Provide a short description of the software being specified and its purpose, including relevant benefits, objectives, and goals.\gnl 1-2 paragraphs describing the scope of the product. Make sure to describe the benefits associated with the product.}
    \subsection{Intended Audience and Document Overview}
        \emph{Describe the different types of reader that the document is intended for, such as developers, project managers, marketing staff, users, testers, and documentation writers (In your case it would probably be the `client' and the professor). Describe what the rest of this SRS contains and how it is organized. Suggest a sequence for reading the document, beginning with the overview sections and proceeding through the sections that are most pertinent to each reader type.}
    \subsection{Definitions, Acronyms, and Abbreviations}
        \emph{Define all the terms necessary to properly interpret the SRS, including acronyms and abbreviations. You may wish to build a separate glossary that spans multiple projects or the entire organization, and just include terms specific to a single project in each SRS.\gnl Please provide a list of all abbreviations and acronyms used in this document sorted in alphabetical order.}
        \gnl The glossary terms can be used as below:
        \\\hspace*{10mm} We looked into hosting our \gls{kubernetes} services on \gls{aws}, \gls{gcp}, \gls{digitalocean}, and \dots \Gls{aws} turned out to be more expensive than \gls{digitalocean}, so \dots
        \\ Then commands to print glossary and acronyms like so: 
        \printnoidxglossary
        \printnoidxglossary[type=acronym]
        \printacronyms
    \subsection{Document Conventions}
        \emph{In general this document follows the IEEE formatting requirements. Use Arial font size 11, or 12 throughout the document for text. Use italics for comments. Document text should be single spaced and maintain the 1” margins found in this template. For Section and Subsection titles please follow the template.\gnl Describe any standards or typographical conventions that were followed when writing this SRS, such as fonts or highlighting that have special significance. Sometimes, it is useful to divide this section to several sections, e.g., Formatting Conventions, Naming Conventions, etc.}
    \subsection{References and Acknowledgements}
        \emph{List any other documents or Web addresses to which this SRS refers. These may include user interface style guides, contracts, standards, system requirements specifications, use case documents, or a vision and scope document.\gnl Use the standard IEEE citation guide for this section.}
        % TODO: Automate citations according to IEEE format
