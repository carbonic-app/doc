
\section{Other Non-functional Requirements}\label{sec:non-functional-requirements}
    \subsection{Performance Requirements}\label{sec:performance-requirements}
        % \emph{If there are performance requirements for the product under various circumstances, state them here and explain their rationale, to help the developers understand the intent and make suitable design choices. Specify the timing relationships for real time systems. Make such requirements as specific as possible. You may need to state performance requirements for individual functional requirements or features.\gnl Provide relevant performance requirements based on the information you collected from the client. For example you can say “1. Any transaction will not take more than 10 seconds, etc\dots}
        For the end-user, the product should be reasonably quick and responsive to user interaction. For example, an instructor loading a student's lesson plan should not need to wait significantly longer than if they were directly using the system our backend is interacting with. We want to have as little overhead as possible to ensure user experience is smooth.
        \par The backend is written with concurrency in mind, and the production will be deployed on \gls{gcp} where the cost of powerful resources is fairly low so speed shouldn't be an issue. However, this does not tolerate terribly optimized code; it should still be developed with efficiency in mind, but not the highest priority. The production environment should scale up to many potentially many thousands of concurrent users with no problem.
    \subsection{Safety and Security Requirements}\label{sec:safety-security}
        % \emph{Specify those requirements that are concerned with possible loss, damage, or harm that could result from the use of the product. Define any safeguards or actions that must be taken, as well as actions that must be prevented. Refer to any external policies or regulations that state safety issues that affect the product’s design or use. Define any safety certifications that must be satisfied. Specify any requirements regarding security or privacy issues surrounding use of the product or protection of the data used or created by the product. Define any user identity authentication requirements.\begin{itemize} \item Provide relevant safety requirements based on your interview with the client or, on your expectation for the product. \item Describe briefly what level of security is expected from this product by your client and provide a bulleted (or numbered) list of the major security requirements. \end{itemize}}
        Users can expect that their information is not shared with other users by our system, as part of the functionality is for communications to be contained within this product. Additionally, since the product relies on \gls{gcp} and security will be kept in mind during development, there is a high security level for the entire product.
        \par Major security requirements are as follows:
        \begin{itemize}
            \item Secure sign-in
            \item Restrict access of users' personal information
            \item Entire system is reasonably secure from outside threats
        \end{itemize}
    \subsection{Software Quality Attributes}\label{sec:quality-attributes}
        % \emph{Specify any additional quality characteristics for the product that will be important to either the customers or the developers. Some to consider are: adaptability, availability, correctness, flexibility, interoperability, maintainability, portability, reliability, reusability, robustness, testability, and usability. Write these to be specific, quantitative, and verifiable when possible. At the least, clarify the relative preferences for various attributes, such as ease of use over ease of learning.\gnl Use subsections (e.g., 4.3.1 Reliability, 4.3.2 Portability, etc\dots) provide requirements related to the different software quality attributes. Base the information you include in these subsections on the material you have learned in the class. Make sure, that you do not just write “This software shall be maintainable…” Indicate how you plan to achieve it, etc.}
        \subsubsection{Reliability}\label{sec:reliability}
            The site itself should have low downtime, ensured by relying on Google's infrastructure and by using \gls{ci}/\gls{cd} to reduce errors reaching the production environment. 
        \subsubsection{Portability}\label{sec:portability}
            While the product does rely on an internet connection to interact with our backend, the user-facing product should be usable on a variety of hardware form factors and \glspl{os}. The backend should also be portable since it is designed with containerization in mind, and can be deployed in \gls{the-cloud} or locally on a developer's machine.
        \subsubsection{Maintainability}\label{sec:maintainability}
            By using techniques such as code reviews, \glspl{vcs}, \gls{ci}/\gls{cd}, and modern programming languages, we can reduce the amount of technical debt during development and therefore make the product more maintainable in the long run. Additionally, documentation thus far is done primarily in \gls{latex} where formatting and referencing are set-and-forget, so adding content is incredibly easy.
