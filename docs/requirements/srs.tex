% This is a new document
\documentclass{article}

% TODO: Modularize using \input, check out https://en.wikibooks.org/wiki/LaTeX/Modular_Documents

% Links in table of contents
\usepackage[hidelinks]{hyperref}

% General page dimensions
\usepackage[letterpaper, top=1in, bottom=1in, left=1in, right=1in]{geometry}

% Fancy page options
\usepackage{fancyhdr}

\usepackage[pdftex]{graphicx}
\usepackage[utf8]{inputenc}
\usepackage{amsfonts}
\usepackage{amssymb}
\usepackage{amsmath}
\usepackage{wasysym}

% Columns if needed
\usepackage{multicol}

% \begin{comment} environment
\usepackage{comment}

% Colors!
\usepackage{color}

% More options for tables
\usepackage{array}

% Code blocks and syntax highlighting
\usepackage{listings}

% Highlighting, \hl{texthere}
\usepackage{soul}

% Subfigures, subtables, etc
\usepackage{subcaption}

% Reference last page number
\usepackage{lastpage}

% Automate definitions, acronyms, and abbreviations
\usepackage[acronym]{glossaries}
\makenoidxglossaries
% \makeglossaries didn't work, see link for details (search "noidx")
% http://tug.ctan.org/tex-archive/macros/latex/contrib/glossaries/glossariesbegin.html

% For special positioning of the WSU-V banner
\usepackage{tikzpagenodes}

% Drawing and graphs
\usepackage{tikz}
\usetikzlibrary{automata, positioning, arrows}
\tikzset{initial text=$ $,
         ->,
         >=stealth',
         node distance=2cm}



% Custom commands
\newcommand{\projectName}{CS320proj}
\newcommand{\projectVersion}{0.1}
\newcommand{\myClass}{CS 320}
\newcommand{\myAssignment}{Software Requirements Specification}
\newcommand{\myDate}{\today}
\newcommand{\mySemester}{WSU-V Fall 2019}

% People
% Garrett
    \newcommand{\gfname}{Garrett}
    \newcommand{\glname}{Walter}
    \newcommand{\gname}{\gfname\ \glname}
    \newcommand{\gidnum}{11681610}
    \newcommand{\gemail}{garrett.walter@wsu.edu}
% Daniel
    \newcommand{\dfname}{Daniel}
    \newcommand{\dlname}{Lee}
    \newcommand{\dname}{\dfname\ \dlname}
    \newcommand{\didnum}{10101010}
    \newcommand{\demail}{something@some.thing}

% Convenient shortcuts
    \newcommand{\gand}{\wedge}             % Logical 'and'
    \newcommand{\gor}{\vee}                % Logical 'or'
    \newcommand{\gnot}{\neg \hspace{.2em}} % Logical 'not'
    \newcommand{\gpipe}{\hspace{.2em} \vert \hspace{.2em}} % Just doing | doesn't leave much space on either side
    \newcommand{\gsum}{\sum\limits}        % Saving keystrokes
    \newcommand{\gnl}{\vspace{1em}\\}      % New line, but with a little extra space

% Title
\title{\vspace{32ex}\Huge\projectName\\\vspace{.5ex}\small Version \projectVersion\\\LARGE\myAssignment\vspace{4ex}}
\author{
    Prepared by:\\
    \begin{tabular}{ r c l }
        \gname & \texttt{\#}\gidnum & \gemail \\
        \dname & \texttt{\#}\didnum & \demail
    \end{tabular}
    \vspace{4ex}
}
\date{\mySemester\\This document compiled on \myDate}

% Common page styling
\pagestyle{fancy}
\lhead{\emph{\myAssignment\ for \projectName}}
\chead{}
\rhead{Page \thepage}
\rhead{Page \thepage\ of \pageref*{LastPage}}
\lfoot{}
\cfoot{} % Empty footer
\rfoot{}

% Line below header and above footer
\renewcommand\headrulewidth{0.4pt}
\renewcommand\footrulewidth{0pt}

% Custom stuff for the first page
\fancypagestyle{firstpage}{
    \fancyhf{}
    \renewcommand\headrulewidth{0.4pt}
    \renewcommand\footrulewidth{0pt}
}

% Very visible TODO
\newcommand{\todo}{{\LARGE\color{red}\textbf{TODO}}}

% Colors for code syntax highlighting
\definecolor{mygreen}{rgb}{0,0.6,0}
\definecolor{mygray}{rgb}{0.5,0.5,0.5}
\definecolor{mymauve}{rgb}{0.58,0,0.82}
\lstset{ 
    basicstyle=\ttfamily\footnotesize, % the size of the fonts that are used for the code
    breakatwhitespace=false,           % sets if automatic breaks should only happen at whitespace
    breaklines=true,                   % sets automatic line breaking
    % titlepos=b,                      % sets the caption-position to bottom
    commentstyle=\color{mygreen},      % comment style
    deletekeywords={...},              % if you want to delete keywords from the given language
    extendedchars=true,                % lets you use non-ASCII characters; for 8-bits encodings only, does not work with UTF-8
    frame=single,	                   % adds a frame around the code
    keepspaces=true,                   % keeps spaces in text, useful for keeping indentation of code (possibly needs columns=flexible)
    keywordstyle=\color{blue},         % keyword style
    language=c,                        % the language of the code
    morekeywords={*,...},              % if you want to add more keywords to the set
    numbers=left,                      % where to put the line-numbers; possible values are (none, left, right)
    numbersep=5pt,                     % how far the line-numbers are from the code
    numberstyle=\tiny\color{mygray},   % the style that is used for the line-numbers
    rulecolor=\color{black},           % if not set, the frame-color may be changed on line-breaks within not-black text (e.g. comments (green here))
    showspaces=false,                  % show spaces everywhere adding particular underscores; it overrides 'showstringspaces'
    showstringspaces=false,            % underline spaces within strings only
    showtabs=false,                    % show tabs within strings adding particular underscores
    stepnumber=1,                      % the step between two line-numbers. If it's 1, each line will be numbered
    stringstyle=\color{mymauve},       % string literal style
    tabsize=4,                         % sets default tabsize
    title=\lstname,                    % show the filename of files included with \lstinputlisting; also try caption instead of title
    backgroundcolor=\color{white}      % choose the background color; you must add \usepackage{color} or \usepackage{xcolor}; should come as last argument
}

% Definitions, acronyms, and abbreviations
\newglossaryentry{kubernetes}{
    name=Kubernetes,
    description={An open-source system for automating deployment, scaling, and management of containerized applications}
}
\newglossaryentry{digitalocean}{
    name={Digital Ocean},
    description={Cloud infrastructure services that help to deploy and scale applications.}
}
\newacronym{aws}{AWS}{Amazon Web Services}
\newacronym{gcp}{GCP}{Google Cloud Platform}



\begin{document}
\pagenumbering{roman}

\begin{titlepage}

% Attempt at getting the WSU-V banner in the top left of title page
% Based on: https://tex.stackexchange.com/a/386331
% TODO: Next step is to take inspiration recreate the title on our own to
%       avoid using the \maketitle command
%       Based on: https://tex.stackexchange.com/a/324152
% \begin{tikzpicture}[remember picture, overlay, shift={(current page.north west)}]
%     \node[anchor=north west, xshift=1cm, yshift=-1cm]{\includegraphics[width=.5\textwidth]{images/WSU-V_Banner.png}};
% \end{tikzpicture}

\maketitle
\thispagestyle{firstpage}
\end{titlepage}

\pagebreak
\tableofcontents
\pagebreak

\pagenumbering{arabic}

\section{Introduction}
    \subsection{Document Purpose}
        \todo
    \subsection{Product Scope}
        \todo
    \subsection{Intended Audience and Document Overview}
        \todo
    \subsection{Definitions, Acronyms, and Abbreviations}
    The glossary terms can be used as below:
    \\\hspace*{10mm} We looked into hosting our \gls{kubernetes} services on \gls{aws}, \gls{gcp}, \gls{digitalocean}, and \dots \Gls{aws} turned out to be more expensive than \gls{digitalocean}, so \dots
        \\ Then commands to print glossary and acronyms like so: 
        \printnoidxglossary
        \printnoidxglossary[type=acronym]
        \printacronyms
    \subsection{Document Conventions}
        \todo
    \subsection{References and Acknowledgements}
        \todo

\pagebreak
\section{Overall Description}
    \subsection{Product Perspective}
        \todo
    \subsection{Product Functionality}
        \todo
    \subsection{Users and Characteristics}
        \todo
    \subsection{Operating Environment}
        \todo
    \subsection{Design and Implementation Constraints}
        \todo
    \subsection{User Documentation}
        \todo
    \subsection{Assumptions and Dependencies}
        \todo
\pagebreak
\section{Specific Requirements}
    \subsection{External Interface Requirements}
        \subsubsection{User Interfaces}
            \todo
        \subsubsection{Hardware Interfaces}
            \todo
        \subsubsection{Software Interfaces}
            \todo
        \subsubsection{Communications Interfaces}
            \todo
    \subsection{Functional Requirements}
        \todo
    \subsection{Behavior Requirements}
        \subsubsection{Use Case View}
            \todo

\pagebreak
\section{Other Non-functional Requirements}
    \subsection{Performance Requirements}
        \todo
    \subsection{Safety and Security Requirements}
        \todo
    \subsection{Software Quality Attributes}
        \todo

\pagebreak
\section{Other Requirements}
    \todo

\pagebreak
% \pagenumbering{alph} % Causes issue with last page, like "Page 4 of a"
% Possible workaround would be manually setting a reference to the last page
%       before the appendix, and using that instead of \pageref*{LastPage}
\appendix

\section{Data Dictionary}
    \todo

\pagebreak
\section{Group Log}
    \todo

% TODO: Automate bibliography according to IEEE format

\end{document}